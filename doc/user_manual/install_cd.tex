\section{Installare Opsip da CD}

\begin{enumerate}
\item Scegliere l'opzione Installare sipXecs

\item Scegliere il network device per l'installazione (tipicamente eth0)

\item Configurare la parte di rete: Disabilitare la parte IPV6 e configurare a mano la parte IPV4

\item Assegnare IP, gateway e nome server

\item Procedere con la normale installazione di CentOS 6 (dati amministrazione e partizionamento)
\end{enumerate}

Al primo accesso viene avviata la configurazione del sistema opsip in cui e' possibile:

\begin{itemize}
\item configurare i parametri di rete. Questa operazione permette di riconfigurare i parametri di rete nel caso in cui ci siano stati errori nella configurazione in fase di installazione (passo 2 della procedura). Tale sezione permette di effettuare la configurazione dei device di rete (assegnare nuovamente IP gateway e server name) e dei DNS (DNS primario secondiario e terziario, nome del dominio e dell'host).

\item installare il server come primo server o facente parte di un cluster gia' esistente (nella configurazione base il server non e' in cluster, quindi scegliamo l'opzione ``y'' ossia ``first server'' )

\item Configurare il SIP domain. Se si sta configurando un server come unico server dell'impianto (non facente parte di un cluster), si puo' utilizzare il valore ``localdomain''. (come indicato anche dalle indicazioni a video)

\item Configurare il SIP realm name. Solitamnete lo stesso del SIP domain configurato al passo precedente.

\item Configurare l'host name. Utilizzare localhost come suggerito da sistema se si tratta dell'unico server dell'impianto e per facilitare l'installazione.

\item Configurare il dominio di rete.
\end{itemize}

A questo punto viene mostrato il dettaglio della cofigurazione risultante, con la possibilita' di cambiarla se qualcosa risultasse errato.
