\section{Installare Opsip da codice sorgente}

Considerando un sistema Centos \textsuperscript{\textregistered} 6 correttamente installato e funzionante procediamo alla installazione del prodotto.

Prima di procedere con ogni altro passo assicurarsi di aver effettuato le fasi di preinstallazione:

\subsection{configurare correttamente l'utente}

Per poter procedere \`e necessario che l'utente abbia i permessi per eseguire comandi privileggiati quindi:

abilitare il gruppo wheel per gli utenti con permessi si superadmin 

\begin{verbatim}
root$ vm /etc/sudoers trovare la linea dove parla di wheel e decommentare
\end{verbatim}

aggiungere il gruppo wheel all'utente di amministrazione 

\begin{verbatim}
usermod -G wheel -a $USER
\end{verbatim}

fare logout e rinetrare in modo che abbia effetto la modifica della lista dei gruppi

\subsection{installazione pacchetti necessari}

I prima battuta fare un completo aggiornamento del sistema

\begin{verbatim}
sudo yum update
\end{verbatim}

Installare vari pacchetti:
\begin{verbatim}
sudo yum install emacs-nox 
\end{verbatim}

\subsection{attivare servizi}
attivare sshd e named(bind)

\begin{verbatim}
$sudo yum install openssh-server bind
$sudo chkconfig sshd on
$sudo service sshd start
$sudo chkconfig named on 
$sudo service named start
\end{verbatim}

Dopo aver installato fedora nella release corretta si puo' procedere con il processo di installazione. Occorrera\` innanzitutto recuperare il codice sorgente dell'intero progetto effettuando i seguenti passi:

\subsection{retrieve codice sorgente}

Recuperare i sorgenti del progetto (questo task richiede circa 90 minuti) 

\begin{verbatim}
$git clone git://github.com/dhubler/sipxecs.git
$cd sipxecs  
$git submodule init 
$git submodule update
$git clone http://git.sip2ser.net/sip2ser
\end{verbatim}

Per installare la corretta versione del prodotto occorre spostarsi sui corrispondenti branch del progetto:

\begin{verbatim}
$git checkout release-4.6
$cd sip2ser
$git checkout opsip-release4.6
\end{verbatim}

Il passo successivo e' quello di intallare i repository aggiuntivi per Centos:

\begin{verbatim}
$sudo yum install epel-release
\end{verbatim}

Tale step creer\`a il file \emph{/etc/yum.repo.d/epel.repo}