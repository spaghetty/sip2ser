\section{Generazione del CD di installazione}

\begin{enumerate}
\item Download della ISO di Centos (esenpio in /home/myuser/Downloads) : 
\begin{verbatim}
wget http://centos.fastbull.org/centos/6.3/isos/x86_64/CentOS-6.3-x86_64-minimal.iso
\end{verbatim}

\item Spostarsi nella directory di build all'interno del tree di sipxecs e  lanciare il configure per il supporto alla generazione del cd: 
\begin{verbatim}
$rake config[ISO] (il quale esegue il comando di configure con opportuni parametri: ../configure --enable-centos-iso ISO_DIR=/home/myuser/Downloads RPM_DIST_DIR=/home/myuser/sipxecs/repo CENTOS_RSYNC_URL=rsync://mirrors.rit.edu)
\end{verbatim}

Si noti che all'interno della directory di build esiste il file 111-[ISO | BUILD] che indica se il configure e' in modalita' di compilazione (BUILD) o modalita' di supporto alla generazione del cd (CD).

\item Lanciare la creazione degli rpm di sipxecs per la distribuzione corretta (CentOS 6 nel nostro caso) se non e' stato gia' fatto:
\begin{verbatim}
$cd build
$make distro.centos-6-x86_64.sipx.rpm
\end{verbatim}

\item Lanciare la creazione degli rpm sip2ser per la distribuzione corretta (CentOS 6 nel nostro caso) se non e' stato gia' fatto:
\begin{verbatim}
$cd build
$make distro.centos-6-x86_64.s2s.rpm
\end{verbatim}

\item Lanciare la generazione del CD: 
\begin{verbatim}
make iso-64
\end{verbatim}
\end{enumerate}